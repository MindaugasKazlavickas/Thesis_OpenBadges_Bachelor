\appendix

\phantomsection
\addcontentsline{toc}{section}{APPENDICES}
\begin{center}
    \Large \textbf{APPENDICES}
\end{center}
\vspace{1cm}

\appendixtitle{Appendix 1. User Survey Questionnaire}{Appendix 1. \textnormal{User Survey Questionnaire}}
\label{appendix:survey}
\vspace{0.5em}

The following appendix contains the complete list of user survey questions used in the study. The lists provide the question to the user, followed by available choices or forms of answers.

Questions were made available in both English and Lithuanian. Rated questions were within a scale of 1-4, with 1 as "Strongly disagree" and 4 as "Strongly agree".
\begin{enumerate}
  \item Choose your preferred language. Choice between English and Lithuanian.

  \textbf{Content and Clarity}
  \item I better understand the concept of Open Badges after completing this website. Rated 1-4.
  \item I understand what the issued badge represents and how it could be used. Rated 1-4.
  \item Which task or activity did you find the most useful or engaging for understanding Open Badges? Why? Open question.
  \item Which task or activity was the least helpful or enjoyable? Why? Open question.
  
  \textbf{Gamification and User Interaction}
  \item The interactive tasks made the learning experience more engaging. Rated 1-4.
  \item Did the interactive elements ever confuse or distract you from the learning content? Yes or No question.
  \item If you answered "Yes" in the previous question, please describe what confused or distracted you, and why. Open question.
  \item Did the progression system (unlocking sections as you go) help or hinder your learning experience? Rated from “Hindered” to “Helped”, with a “Not sure” and “Had no effect” as extra options.
  \item The score system engaged me to learn more. Rated 1-4.
  \item Receiving a badge at the end felt like a meaningful reward. Rated 1-4.

  \textbf{Usability and General Experience}
  \item The website is enjoyable to use overall. Rated 1-4.
  \item The website was easy to navigate and worked well on my device. Rated 1-4.
  \item I would recommend this site to someone interested in learning about Open Badges. Rated 1-4.
  \item Any other comments, suggestions, or technical issues you encountered? Open question.
\end{enumerate}

Below provided is the Lithuanian version of the questionnaire:

\begin{enumerate}
\item Pasirinkite kalbą. Pasirinkimas tarp lietuvių ir anglų kalbų.

\textbf{Turinys ir Aiškumas}
\item Aš geriau suprantu Atvirųjų Ženkliukų koncepciją po šios svetainės peržiūros. Vertinimas nuo 1 iki 4.
\item Suprantu, ką reiškia suteiktas ženkliukas ir kaip jis gali būti naudojamas. Vertinimas nuo 1 iki 4.
\item Kuri užduotis ar veikla jums buvo naudingiausia arba įdomiausia svetainėje mokantis apie Atviruosius Ženkliukus? Kodėl? Atviras klausimas.
\item Kuri užduotis ar veikla jums buvo mažiausiai naudinga arba įdomi? Kodėl? Atviras klausimas.

\textbf{Žaidybinimas ir Vartotojo Sąveika}
\item Interaktyvios užduotys padarė mokymosi patirtį įdomesnę. Vertinimas nuo 1 iki 4.
\item Ar interaktyvūs elementai kada nors jus supainiojo ar atitraukė nuo mokymosi turinio? Taip / Ne klausimas.
\item Jei atsakėte „Taip“ ankstesniame klausime, aprašykite, kas jus supainiojo ar atitraukė, ir kodėl. Atviras klausimas.
\item Ar progresavimo sistema (skyrių atrakinimas vienas po kito) padėjo ar trukdė jūsų mokymosi patirčiai? Vertinimas nuo „Trukdė“ iki „Padėjo“, su „Nesu tikras“ ir „Neturėjo įtakos“ papildomais variantais.
\item Taškų sistema paskatino mane mokytis. Vertinimas nuo 1 iki 4.
\item Ženkliuko gavimas pabaigoje buvo prasmingas apdovanojimas. Vertinimas nuo 1 iki 4.

\textbf{Naudojimo Patogumas ir Bendra Patirtis}
\item Naudotis šia svetaine buvo malonu. Vertinimas nuo 1 iki 4.
\item Svetainė buvo lengvai naršoma ir tinkamai veikė mano įrenginyje. Vertinimas nuo 1 iki 4.
\item Rekomenduočiau šią svetainę žmogui, norinčiam sužinoti daugiau apie Atvirus Ženkliukus. Vertinimas nuo 1 iki 4.
\item Turite kitų komentarų, pasiūlymų ar pastebėjote techninių problemų? Atviras klausimas.
\end{enumerate}

\newpage
\appendixtitle{Appendix 2. External Resources}{Appendix 2. \textnormal{External Resources}}
\label{appendix:external_resources}

The following appendix contains the complete list of links related to the project and its transparency.

\vspace{0.5em}
\begin{itemize}
  \item Prototype website link: \url{https://atvirieji-zenkliukai.netlify.app/}
  \item Backend API endpoint: \url{https://frontend-openbadges-bachelor.onrender.com/}
  \item Backend list of user sessions: \url{https://frontend-openbadges-bachelor.onrender.com/api/sessions}
  \item Individual user session results. Please replace "sessionId" with any session ID from the previous link: \url{https://frontend-openbadges-bachelor.onrender.com/api/logs/sessionId}
  \item Full source code: \url{https://github.com/MindaugasKazlavickas/Frontend_OpenBadges_Bachelor}
  \item This LaTeX thesis: \url{https://github.com/MindaugasKazlavickas/Thesis_OpenBadges_Bachelor}
\end{itemize}

\textit{Note: The hosted websites on Netlify and Render are intended to be maintained and freely available online until at least June 2026.}

\textit{Note: Badge issuance tied to Open Badge Factory is a paid service and will only be supported until mid-June of 2025. Afterwards attempted badge requests will result a "Something went wrong" error 500.}