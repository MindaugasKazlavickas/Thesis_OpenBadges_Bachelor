\section*{CONCLUSIONS}
\addcontentsline{toc}{section}{CONCLUSIONS}
\begin{enumerate}
    \item \textbf{Theoretical Foundations} The analysis of gamification and open badge theory provided a strong foundation for the platform’s design. 
    The implementation successfully applied key motivational principles of \acrshort{sdt}, flow theory and \acrshort{mda}, leading to high user engagement and positive feedback in both structured tasks and open-ended responses. 
    This theoretical grounding was essential in ensuring that gamification elements like score feedback, progression, and reward systems were meaningful and not superficial.
    \item \textbf{Requirements and Design Specification} The functional and technical requirements identified early in the process, such as a mobile-first layout, structured task progression, and minimal cognitive friction, proved effective. 
    The feedback and performance data indicate that the site's architecture, section flow, and interaction model met usability and accessibility expectations, particularly for the intended higher education student audience.
    \item \textbf{Platform Development} The developed prototype successfully incorporated interactive, gamified elements that educated users about Open Badges. 
    The sequential task design consistently improved badge literacy in a clear and engaging way, with small, gamified tasks of card sorting, item selection and card association showing good learning impact. 
    The scenario identification task revealed issues with clarity. 
    The overall structure demonstrated that gamification could support and even enhance educational messaging when implemented thoughtfully.
    \item \textbf{Testing and Evaluation} The evaluation confirmed that the platform achieved its learning goals, with strong self-reported user understanding and enjoyment. 
    The mixed performance results and feedback highlighted specific areas for refinement, most notably the clarity and structure of task 3 and badge value perception. 
    Despite technical setbacks and a limited sample size, the testing methodology yielded actionable insights, validating both the concept and its execution while identifying concrete paths for improvement in future iterations and research.
\end{enumerate}

\newpage