\section*{INTRODUCTION}
\addcontentsline{toc}{section}{INTRODUCTION}
%
\textbf{Relevance of the topic.} The increasing focus on holistic education highlights the significance of recognising extracurricular achievements alongside academic credentials outside of the academic environment. 
Digital badges provide a flexible system for tracking and validating non-formal learning, offering students, staff and other participants an opportunity to articulate their skills effectively. 
Unfortunately, existing implementations fall short of engaging users and demonstrating the relevance to industry recruiters and evaluators who are responsible for incorporating these badges into industry practices.

Research \cite {Stefaniak_and_Carey} suggests the potential for open badges to foster confidence. 
Similarly, research \cite{definition}, \cite{nicholson2015} further advocates for meaningful gamification as a mechanism to enhance user involvement and long-term motivation with these systems to lead to lasting personal benefits.

By analyzing the environment of the application, and addressing gaps in awareness to improve engagement and practical implementation, this project aims to contribute towards gamified educational tools and their role in fostering self-determination, lifelong learning and further recognition of gamified systems. 
The project is concerned with the Vilnius Tech Open Badge System and for the purposes of the work, the core audience from this point will be considered higher education students.

\textbf{Problem -} The current systems for tracking extracurricular achievements lack robust user engagement strategies and fail to appeal to industry specialists as a form of credibility.
This disengagement often leads to user fatigue, frustration, and underutilization \cite {FunaAndAaron}. 
Consequently, some students who may struggle to recognize and articulate their skills are limiting their professional growth due to this underutilization.
Additionally, due to limited awareness, and recognition struggles \cite {federalIssues}, it is difficult to present them as appealing alternatives to recruiters.

\textbf{Research Object -} A gamified educational website-browser game that integrates a digital badge system to promote the recognition of open badges.

\textbf{Aim -} To develop an interactive, user-friendly educational website-browser game that uses gamification to showcase the value of open badges, facilitating recognition within the industry and aiding in the articulation of users’ skills and achievements.

\textbf{Tasks:} 
\begin{enumerate}
  \addtolength{\itemsep}{-0.5\baselineskip} 
  \item Analyze theoretical foundations of digital badges, gamification, and their application in education to identify best practices.
  \item Define technical and functional requirements for a gamified educational platform tailored to user engagement.
  \item Develop a prototype platform incorporating gamified elements that educate users about the badge system and its applications.
  \item Test and evaluate the prototype to assess user engagement and understanding of the badge system.
\end{enumerate}

\textbf{Research Methods:}
\begin{enumerate}
  \addtolength{\itemsep}{-0.5\baselineskip} 
  \item Literature Review: Conduct a focused review of existing research on gamification, digital badge systems, and their use in educational contexts to inform the platform's design, both from a theoretical and technical perspective.
  \item Prototype Development: Utilize the Phaser 3 game engine and modern web technologies to create an interactive and scalable platform.
  \item Evaluation: Explore testing possibilities, applications, and suggest future research and developments within the fields of gamification.
\end{enumerate}

Notably, this thesis is an extension of a previously completed academic report \cite{me2024}, "Gamification in Web Development".
\newpage