\phantomsection
\section*{INTRODUCTION}
\addcontentsline{toc}{section}{INTRODUCTION}
%
\textbf{Relevance of the topic.} The increasing focus on holistic education highlights the significance of recognising extracurricular achievements alongside academic credentials outside of the academic environment. 
Digital badges provide a flexible system for tracking and validating non-formal learning, offering students an opportunity to articulate their skills effectively. 
Unfortunately, existing implementations fall short of engaging users and demonstrating the relevance to industry recruiters and evaluators who are responsible for incorporating these badges into industry practices.

Research (\cite {Stefaniak_and_Carey}) suggests the potential for open badges to foster confidence. 
Similarly, research (\cite{definition}), (\cite{nicholson2015}) further advocates for meaningful gamification as a mechanism to enhance user involvement and long-term motivation with these systems to lead to lasting personal benefits.

By analyzing the environment of the application, and addressing gaps in awareness to improve engagement and practical implementation, this project aims to contribute towards gamified educational tools and their role in fostering self-determination, lifelong learning and further recognition of gamified systems. 
The project is concerned with the Vilnius Tech Open Badge System and for the purposes of the work, the core audience from this point will be considered higher education students.

\textbf{Problem -} The current systems for tracking extracurricular achievements lack robust user engagement strategies and fail to appeal to industry specialists as a form of credibility.
This disengagement often leads to user fatigue, frustration, and underutilization (\cite {FunaAndAaron}). 
Consequently, some students who may struggle to recognise and articulate their skills are limiting their professional growth due to this underutilization.
Additionally, due to limited awareness, and recognition struggles (\cite {federalIssues}), it is difficult to present them as appealing alternatives to recruiters.

\textbf{Research Object -} A gamified educational website that integrates a digital badge system to promote the recognition of open badges.

\textbf{Aim -} To develop an interactive educational website that applies gamification principles to help users understand the purpose, functioning and value of open digital badges in the context of competence-based learning recognition.

\textbf{Tasks:} 
\begin{enumerate}
  \addtolength{\itemsep}{-0.5\baselineskip} 
  \item Conduct a theoretical analysis of open digital badges, their relevance in learning recognition, and the application of gamification principles across educational and non-educational contexts to inform learning recognition strategies.
  \item Develop a comparative review and design strategy based on gamification theory, pedagogical models, and layout and interface decisions to support user engagement and effective learning.
  \item Program a gamified educational website using selected technologies, integrating task-based learning, progression systems, and open badge issuance in compliance with metadata standards.
  \item Evaluate the website’s functionality and educational effectiveness through task performance analysis and structured user feedback.
\end{enumerate}

\textbf{Result:} An interactive, gamified educational website was created to introduce users to open digital badges, their operating principles, and their benefits in the context of competence recognition.

\textbf{Research Methods:}
\begin{enumerate}
  \addtolength{\itemsep}{-0.5\baselineskip} 
  \item Literature Review: Conduct a focused review of existing research on gamification, digital badge systems, and their use in educational contexts to inform the platform's design, both from a theoretical and technical perspective.
  \item Website Development: Utilize the React.js framework to create an interactive, scalable and easily integrateable platform.
  \item Evaluation: Explore and apply testing possibilities, applications, and suggest future research and developments within the fields of gamification.
\end{enumerate}

This thesis is an extension upon a previously completed academic report on gamification in web development (\cite{me2024}). 
That report is not publicly available due to confidentiality. 
The topic of this thesis was coordinated with the supervisor to allow continued research based on the foundations and insights established in that earlier work.
\newpage