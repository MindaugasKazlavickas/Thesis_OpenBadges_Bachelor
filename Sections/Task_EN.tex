{\fontsize{10}{12}\selectfont
\begin{center}
\scshape VILNIUS GEDIMINAS TECHNICAL UNIVERSITY\\
\scshape FACULTY OF FUNDAMENTAL SCIENCES\\
\scshape DEPARTMENT OF GRAPHICAL SYSTEMS\\[2.0\baselineskip]
\end{center}
%
\noindent
\begin{tabularx}{\textwidth}{Xc}
\begin{minipage}[t]{0.8\textwidth}
    {Study field: Informatics Engineering}\\
    {Study programme: Multimedia and Computer Design, state code 6121BX025}\\
    {Specialisation: Multimedia and Computer Design}
\end{minipage}
\vspace{1cm}

&
\begin{minipage}[t]{0.2\textwidth}
    {APPROVED BY}\\
    {Head of Department}\\
    {Romualdas Baušys}\\
    {2025-05-27}
\end{minipage}
\end{tabularx}
%
\vspace{4pt}
%
\begin{center}
{\textbf {OBJECTIVES FOR BACHELOR'S DEGREE FINAL WORK}}\\[0.5\baselineskip]
{No. MKDf-21/2–11243}\\
{Vilnius}\\
\end{center}
%
\noindent
{Student: Mindaugas Kazlavickas}\\[0.5\baselineskip]
%
\noindent
{ Title of the final work (project): Design of educational website on learning recognition by open badges applying gamification principles}\\[0.5\baselineskip]
%
\noindent
{The final work (project) must be completed in accordance with the academic calendar.}\\[1.0\baselineskip]
\noindent
{THE OBJECTIVES OF THE FINAL WORK (PROJECT):}\\[0.5\baselineskip]
{Aim: To develop an interactive educational website applying gamification principles, aimed at helping users understand the essence of open digital badges, their operating principles, and their value in the context of competence recognition.}\\[1.0\baselineskip]
Tasks:\\[-0.2\baselineskip]
1. Conduct a situational analysis of open digital badges and their value in the context of learning recognition.\\[-0.2\baselineskip]
2. Using gamification principles, create a prototype of an educational website designed to reveal the value of open badges.\\[-0.2\baselineskip]
3. Select appropriate technologies and develop an interactive, gamified educational website about open digital badges.\\[-0.2\baselineskip]
4. Evaluate the functionality and usefulness of the gamified educational website through task performance analysis and a structured user survey.\\[0.5\baselineskip]
\noindent
{Planned results: An interactive, gamified educational website will be developed to introduce users to open digital badges, their operating principles, and their benefits in the context of competence recognition.}\\[1.2\baselineskip]
\noindent
{Academic supervisor Associate Professor Dr. Ingrida Leščauskienė}
}