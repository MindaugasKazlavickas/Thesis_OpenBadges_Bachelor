{\fontsize{10}{12}\selectfont
\begin{center}
\scshape VILNIAUS GEDIMINO TECHNIKOS UNIVERSITETAS \\
\scshape FUNDAMENTINIŲ MOKSLŲ FAKULTETAS \\
\scshape GRAFINIŲ SISTEMŲ KATEDRA\\[2.0\baselineskip]
\end{center}
%
\noindent
\begin{tabularx}{\textwidth}{Xc}
\begin{minipage}[t]{0.82\textwidth}
Studijų kryptis: Informatikos inžinerija\\
Studijų programa: Multimedija ir kompiuterinis dizainas, valstybinis kodas 6121BX025\\
Specializacija: Multimedija ir kompiuterinis dizainas
\end{minipage}
\vspace{1cm}

&
\begin{minipage}[t]{0.18\textwidth}
    {TVIRTINU}\\
    {Katedros vedėjas }\\
    {Romualdas Baušys}\\
    {2025-05-27}
\end{minipage}
\end{tabularx}
%
\vspace{4pt}
%
\begin{center}
{\textbf {BAKALAURO BAIGIAMOJO DARBO (PROJEKTO) UŽDUOTIS}}\\[0.5\baselineskip]
{Nr. MKDf-21/2–11243}\\
{Vilnius}\\
\end{center}
%
\noindent
{Studentas (-ė): Mindaugas Kazlavickas}\\[0.5\baselineskip]
%
\noindent
{Baigiamojo darbo (projekto) tema: Mokomojo tinklalapio apie mokymosi pripažinimą atviraisiais ženkliukais projektavimas taikant sužaidybinimo pricipus.}\\[0.5\baselineskip]
%
\noindent
{Baigiamojo darbo (projekto) užbaigimo terminas pagal numatytą studijų kalendorinį grafiką.}\\[1.0\baselineskip]
\noindent
{BAIGIAMOJO DARBO (PROJEKTO) UŽDUOTIS:}\\[0.5\baselineskip]
{Tikslas: Taikant sužaidybinimo principus, sukurti interaktyvų mokomąjį tinklalapį, skirtą padėti vartotojams suprasti atvirųjų skaitmeninių ženkliukų esmę, veikimo principus bei jų vertę kompetencijų pripažinimo kontekste.}\\[1.0\baselineskip]
Uždaviniai:\\[-0.2\baselineskip]
1. Atlikti situacijos analizę apie atviruosius skaitmeninius ženkliukus ir jų vertę mokymosi pripažinimo kontekste.\\[-0.2\baselineskip]
2. Taikant sužaidybinimo principus, sukurti mokomojo tinklalapio, skirto atskleisti atvirųjų ženkliukų vertę, prototipą.\\[-0.2\baselineskip]
3. Parinkti tinkamas technologijas ir suprogramuoti interaktyvų, sužaidybintą mokomąjį tinklalapį apie atviruosius skaitmeninius ženkliukus.\\[-0.2\baselineskip]
4. Įvertinti sužaidybinto mokomojo tinklalapio funkcionalumą ir naudingumą, taikant užduočių atlikimo analizę ir struktūrizuotą naudotojų apklausą.\\[0.5\baselineskip]
\noindent
{Planuojami rezultatai: Sukurtas interaktyvus, sužaidybintas mokomasis interneto tinklalapis, skirtas vartotojams supažindinti su atviraisiais skaitmeniniais ženkliukais, jų veikimo principais ir nauda kompetencijų pripažinimo kontekste.}\\[1.2\baselineskip]
\noindent
{Vadovas docentas Ingrida Leščauskienė}
}